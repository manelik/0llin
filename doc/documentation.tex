
%%% Local Variables: 
%%% mode: latex
%%% TeX-master: t
%%% End: 




\documentclass[12pt,fleqn,b5paper]{article}

% Load packages for symbols and figures.

\usepackage[utf8]{inputenc} % Caracteres con acentos.
\usepackage[spanish]{babel} % Títulos en español
\usepackage{latexsym}       % Símbolos
\usepackage{amsmath}  
\usepackage{amssymb}        %Soporte para símbolos y fondos matemáticos
\usepackage[pdftex]{color,graphicx} % FIGURAS PNG
\usepackage{epsfig}       % .eps figures support
\usepackage{epstopdf}     % to be able to run with pdflatex, requires                                                                                         
                          % -shell-escape flag to automatically convert                                                                                       
                          % eps figures to pdf in subfolders 
\usepackage[pdftex]{hyperref}
\usepackage{fix-cm}

% Page settings.

%\setlength{\textwidth}{170mm}
%\setlength{\oddsidemargin}{-5mm}
%\setlength{\evensidemargin}{-5mm}


%%%%%%%%%%%%%%%%%%%%%%%%%%
%%%   BEGIN DOCUMENT   %%%
%%%%%%%%%%%%%%%%%%%%%%%%%%

\begin{document}

% Reference equations with section number.

\renewcommand{\theequation}{\thesection.\arabic{equation}}

\parindent 0mm


%%%%%%%%%%%%%%%%%
%%%   TITLE   %%%
%%%%%%%%%%%%%%%%%

\title{C\'odigo \texttt{0llin}.  User's manual}

\author{Jos\'e Manuel Torres \\
Instituto de Ciencias Nucleares, UNAM \\
malcubi@nucleares.unam.mx}

\date{Month, Year}

\maketitle

\tableofcontents


%%%%%%%%%%%%%%%%%%%%%%%%
%%%   INTRODUCTION   %%%
%%%%%%%%%%%%%%%%%%%%%%%%

\pagebreak

\section{Introducci\'on}

El c\'odigo \texttt{0llin} resuelve las ecuaciones de Einstein para
espacios homogéneos e isotrópicos. El objetivo de este es
principalmente didáctico.


%%%%%%%%%%%%%%%%%%%%
%%%   DOWNLOAD   %%%
%%%%%%%%%%%%%%%%%%%%

\section{Descarga del c\'odigo}

Si están leyendo esto probablemente ya consiguieron el c\'odigo. De no
ser así el código por el momento es accesible en el repositorio libre\\

\texttt{\footnotesize 
  https://github.com/manelik/0llin} \\

donde se encuentran opciones para descargar la branch ``master'' como
archivo comprimido .zip o bien hacer un ``fork'' del proyecto en otras
cuentas de github. Tambien es posible ``clonar'' el proyecto si se
tiene {\bf git} previamente configurado\\

\texttt{\footnotesize git clone 
  git@github.com:manelik/0llin.git} \\

{\bf NOTA: el primer caracter del nombre del código es el número
  cero!}

%%%%%%%%%%%%%%%%%%%%%%%
%%%   DIRECTORIES   %%%
%%%%%%%%%%%%%%%%%%%%%%%

\section{Estructura de Archivos}

Al descargar y descomprimir el código, sea cual sea el caso, el
directorio raíz del código contiene las carpetas \texttt{/doc, /src}
en las que se encuentran esta documentacion y las rutinas escritas en
FORTRAN 90 respectivamente. Del mismo modo en el directorio raíz se
encuentra el archivo \texttt{Makefile} que sirve para compilar el
código. Para compilar con {\bf gfortran} basta correr el comando\\

\texttt{make} \\

desde el directorio raíz. En caso de usar otro compilador basta con
editar la variable \texttt{FC} en el archivo Makefile. Por el momento
no se han añadido banderas de optimización.

Una vez compilado el ejecutable se genera en el directorio raíz y
desde aquí basta con correr el comando\\

\texttt{./0llin} \\

Al inicio el código obtiene los parámetros directamente de la terminal
pero es posible dar en su lugar un archivo con dichos parametros en el
mismo orden en que se piden utilizando el redireccionador apropiado\\

\texttt{./0llin < archivo.par} \\

El código siempre genera un archivo de parámetros válido con el nombre
del directorio de salida, para poder reproducir la simulación.



%%%%%%%%%%%%%%%%%%%%%%%%
%%%   OUTPUT FILES   %%%
%%%%%%%%%%%%%%%%%%%%%%%%

\section{Output files}

Los archivos de salida generados por el código se generan en el
directorio especificado durante la ejecución. Estos son archivos con
extensión .tl que contienen los valores guardados como texto en pares \\

\texttt{time    F(time)}
\\



%%%%%%%%%%%%%%%%%%%%%
%%%   SPACETIME   %%%
%%%%%%%%%%%%%%%%%%%%%

\setcounter{equation}{0}
\section{Spacetime and evolution equations}

La métrica en el código se asume de la forma
\begin{equation}
ds^2 = -\alpha(t)^{2} dt^2 + e^{4\phi(t)}{\hat\gamma}_{ij} dx^i dx^j  ,
\end{equation}
con $\alpha(t)$ el lapso, $e^{\phi(t)}$ el factor conforme y
${\hat\gamma}_{ij}$ la métrica espacial conforme. La métrica física
conforme es entonces $\gamma_{ij}=e^{4\phi(t)}{\hat\gamma}_{ij}$ Para
espacios homogéneos e isotrópicos estas funciones dependen unicamente
de $t$, y la métrica conforme se mantiene congelada en su valor
inicial\footnote{Esto se debe a que el único grado de libertad que
  posee un tensor simétrico de rango 2 es la traza, y las ecuaciones
  de evolución mantienen esta forma. }. Un resultado particular es que
el escalar de Ricci de las secciones a $t$ constante toma la forma
\begin{equation}
  \label{eq:Ricci}
  R= e^{-4\phi}\hat R = 6 k e^{-4\phi} \;,\qquad k:= \hat R/6\;.
\end{equation}
La constante $k$ corresponde con la que usualmente aparece en el
denominador de la componente radial de la métrica de
Friedmann-Robertson-Walker escrita en coordenadas esféricas comoviles,
y sin pérdida de generalidad puede tomar los valores $\{0,+1,-1\}$.

\subsection{Evolution Equations}

Bajo estas simetrías las secciones espaciales quedan caracterizadas
por el factor conforme, o equivalentemente $\phi(t)$ y la traza de la
curvatura extrínseca $K$ que mide la expansión de las geodésicas
normales a las hipersuperficies. Sus ecuaciones de evolución están
dadas por
\begin{eqnarray}
  \partial_t \phi &=& - \frac{1}{6} \alpha K \,, \\
  \label{eq:sphere-chidot}
  \partial_t K &=&   \frac{1}{3}\alpha K^2                                                                                           
  + 4 \pi \alpha \left( \rho + S\right) \,, 
  \label{eq:sphere-Kdot}
\end{eqnarray}
con los términos de materia
\begin{eqnarray}
  \label{eq:matter}
  \rho&:=& n^\mu n^\nu T_{\mu\nu}, \\
  S&:=&\gamma^{\mu\nu}T_{\mu\nu}\;. 
\end{eqnarray}
La única constricción no trivial en este caso es la Hamiltoniana, que
toma la forma
\begin{equation}
  \label{eq:Ham}
  \mathcal{H}:= 6ke^{-4\phi(t)}+\frac{2}{3}K(t)^2 -16\pi\rho(t)=0\;.
\end{equation}

\setcounter{equation}{0}
\section{Matter}
\label{sec:matter}
En esta sección se enlistan los tipos de materia incluidos, sus
contribuciones a las fuentes de la geometría y sus ecuaciones de evolución.

\subsection{Constante cosmológica}
\label{sec:lambda}

Tensor de energía momento
\begin{equation}
  {T^\Lambda}_{\mu\nu} =  -\frac{\Lambda}{8\pi}g_{\mu\nu} \,.
\end{equation}
Proyecciones ADM
\begin{eqnarray}
  \label{eq:lambdaTmunu}
  \rho^\Lambda &=&  \frac{\Lambda}{8\pi}\,,\\
  S^\Lambda &=&  -3 \frac{\Lambda}{8\pi}\,.
\end{eqnarray}

La constante cosmológica no es dinámica y sus contribuciones son
iguales a todo tiempo

\subsection{Polvo}

Tensor de energía momento
\begin{equation}
  {T^{\rm dust}}_{\mu\nu} =  \rho_0 u^\mu u^\nu .
\end{equation}
En cosmología consideramos el fluído en reposo $u^\mu=n^\mu$.

Proyecciones ADM
\begin{eqnarray}
  \label{eq:dustTmunu}
  \rho^{\rm dust} &=&  \rho_0 \,,\\
  S^{\rm dust} &=&  0\,.
\end{eqnarray}

La ecuación de evolución se sigue de la conservación de partículas o
de energía/momento de manera equivalente.

\begin{equation}
  \partial_t \rho_0 = \alpha K \rho_0\;. 
\end{equation}

\subsection{Campo escalar}

Tensor de energía momento
\begin{equation}
  {T^{\rm sf}}_{\mu\nu} =  \nabla_\mu \varphi \nabla_\nu \varphi 
  -\frac{1}{2} g_{\mu\nu}\left(\nabla^\sigma \varphi \nabla_\sigma \varphi + m^2\varphi^2\right)
\end{equation}
Conviene introducir una variable auxiliar.
\begin{equation}
  \Pi:= n^\mu \nabla_\mu\varphi\;.
\end{equation}
Entonces las proyecciones ADM quedan
\begin{eqnarray}
  \label{eq:dustTmunu}
  \rho^{\rm sf} &=&  \frac{1}{2}\left(\Pi^2 + m^2\varphi^2\right) \,,\\
  S^{\rm sf} &=&  \frac{3}{2}\left(\Pi^2 - m^2\varphi^2\right)\,.
\end{eqnarray}

El campo escalar evoluciona de acuerdo a la ecuación de Klein-Gordon
\begin{equation}
  \Box \varphi - m^2\varphi = 0\;,
\end{equation}
que puede reescribirse a primer orden como
\begin{eqnarray}
  \partial_t \varphi &=& \alpha \Pi \;, \\
  \partial_t \Pi & =& \alpha \Pi K -m^2\alpha \varphi\;.
\end{eqnarray}


\end{document}